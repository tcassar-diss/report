\section{Design} \label{sec:design}

\af is complex systems software and, therefore, requires a careful
design phase. In a systems security project, added code and complexity
represent more chances for vulnerabilities to appear. Therefore, it was
important to design \af to be \textit{as minimal as possible} while
fulfilling the promise of the proposed fine-grained filtering.

We start by defining a threat model and let primary requirements and
engineering assumptions follow. The project's nature also requires some auxiliary
tooling, which we discuss after we present \af's design. 

\subsection{Threat Model, Objectives, and Assumptions}\label{subsec:objs-assumpions-tm}

% Threat Model

Designing a concrete implementation of any security tooling requires a threat
model. We begin by laying out the situations and preconditions where \af effectively
protects applications. We establish this model based on the separation
between the untrusted application space and the trusted kernel. 

\af assumes 1) a form of \textit{software compartmentalisation} is implemented
on the host system, which will confine an attacker to the compartment 
they have breached. A compartmentalised application comprises multiple \textit
{security domains} which are mutually distrusting \cite{SOK}. Our
implementation of \af assumes compartment boundaries to be the \textit{shared
libraries} in a process's \ac{vma} as in Figure~\ref{fig:compartments-around-vma}. However, the concept is generalisable across the \ac{vma}.

\begin{figure}[ht]
    \centering
    \includegraphics[width=0.96 \linewidth]{./diagrams/compartments-around-so.drawio.pdf} 
    \caption{A diagram showing how compartment boundaries must align
    with file-backed regions of the process's \ac{vma}.}
    \label{fig:compartments-around-vma}
\end{figure}

We assume that 2) an attacker has compromised one or more of these compartments and that
3) they have \textit{arbitrary code execution} within the compartment's context. We
assume 4) the attacker's goal is to exploit a bug in the kernel via a series of
system calls; this could be privilege escalation (e.g. taking over the kernel's
execution flow), leaking/tampering with kernel-critical data, or attempting to
bypass compartmentalisation. 

For completeness, we assume 5) that the attacker does not already have a 
kernel-level exploit and that the process they compromised does not have root
permissions. These scenarios are out of scope as a kernel-level exploit effectively gives the
attacker root access to the host, which invariably results in complete system compromise.

% end Threat Model

% Objectives

Having defined a threat model, we can define and list our objectives.

Our primary aim is to design a system that detects when an application
makes a disallowed system call. Our solution, \af,  must locate the call's origin
within the \ac{vma} and confirm that this memory region has the
appropriate permissions to make the call.

If \af detects a disallowed system call, it must intervene in one of three user
configurable ways: it must kill the offending process, warn the user, or
kill every process currently protected by the filter. These options allow the user to
choose the trade-off between availability and data integrity/confidentiality on
a per-application basis.

We expect to see a significant reduction in the set of system calls that an
attacker can use to compromise an application without detrimentally impacting
application performance. 

System call literature has not defined a standard for measuring the privilege
of a set of system calls -- therefore, we look to propose our own
\textit{probabilistic} metric to quantify the privilege reduction \af
provides over seccomp.

To summarise:

\begin{enumerate}
    \item \af \textit{shall} associate a distinct system call whitelist with each unique file-backed region within a process's \ac{vma}.
    \item \af \textit{shall} \textbf{identify the specific memory region} within the \ac{vma} from which a system call originates. If the invoked system call is not included in the whitelist associated with that originating region, \af \textit{shall} \textbf{trigger an intervention}.
    \item The intervention action taken by \af upon detecting a non-whitelisted system call \textit{shall} be \textbf{configurable by the user}. The available configuration options \textit{shall} include, at minimum, (a) logging a warning, (b) terminating the process that issued the disallowed system call, and (c) terminating all processes being monitored by \af.
    \item \af \textit{shall} demonstrably achieve a \textbf{greater reduction in
        process privileges} compared to a standard Linux seccomp filter
        configuration. We aim to achieve this security enhancement while
        incurring an acceptable runtime performance overhead.
    \item The privilege reduction afforded by \af \textit{shall} be able to be
        \textbf{quantified
        with a newly proposed metric}.
\end{enumerate}

Given the novelty of this solution, no tooling exists that will generate
per-\ac{vma} whitelists. Therefore, we must implement our own \textit{automated whitelist 
generation tool}, \texttt{afgen}. Whitelist generation is a separate research problem and is
orthogonal to our proposed solution but is needed for development. 
Similarly, we need tooling to identify an application's system calls for
evaluation purposes. For this, we implement \texttt{syso}.

% end Objectives

Alongside our aims, we also make some \textit{simplifying assumptions}, 
reserved for rare cases requiring significant added complexity. These 
assumptions are primarily to avoid adding unnecessary code to
\af's design. 
 
We assume that \acg{libc} position in the \ac{vma} is fixed across the process's
lifetime. It is technically possible for \acg{libc} position to change but is
rarely seen in production and would require significant added complexity to
address.

We take care to ensure that this assumption does not introduce a
security risk in Section~\ref{subsubsec:impl-find-site-opt}. 
System calls made in the case where \ac{libc} has been changed are
treated as disallowed system calls.

We also assume that the \texttt{\_\_code\_\_} section of the \ac{libc}
executable is mapped contiguously in memory. Again, this is
true (almost) without exception\footnote{Custom patches to the kernel could
violate this invariant} \cite{glibc-dl-map-segments-h} and allows \af to determine if a given system 
call instruction was made from within \ac{libc} quickly.

It is also important to note that this assumption does not introduce a security
risk due to the way that system call site identification is implemented and shown in
Section \ref{subsubsec:impl-find-site-opt} .

The obvious limitation of this design is that it only makes sense for
dynamically linked executables. Statically linked executables do not make use of
shared libraries, and therefore, making filters for each shared library does not
make sense.

With the relevant background fully explained and the requirements defined, we
now present the design of \af. 

\subsection{Architecture} \label{subsec:arch}

\begin{figure}[ht]
    \centering
    \includegraphics[width=0.8 \linewidth]{./diagrams/high-level-overview.drawio.pdf} 
    \caption{High level overview of \afg architecture}
    \label{fig:arch-overview}
\end{figure}

\af is comprised of a (kernel space) backend and a (user space) frontend.
\ac{bpf} maps and ringbuffers are used to allow the backend and frontend to
communicate. The frontend is responsible for spawning the filtered application,
handling configuration, and \textit{attaching} the \ac{bpf} program to its
tracepoint (discussed further in Section \ref{sec:implementation}).

The backend of \af is made up of a single \ac{bpf} program and a selection of \ac{bpf}
maps/ringbuffers to communicate with user space.
The program is attached to a \textit{raw tracepoint} and runs every time
\textit{any process} makes a system call.

\subsubsection{Determining Which System Calls to
Filter}\label{subsec:design-fork-following}

This means \af needs a way to identify which system calls were made by the
protected app. To do this, we store the \acp{pid} of filtered processes
in \iac{bpf} map. Then, on every system call, we check if the
called process's \ac{pid} is in the map of filtered processes. If not, we ignore
the system call. Otherwise, we continue with the filtering.

We also want to apply the filter to any child processes the
application might create - this is called \textit{fork following}. The reason for this is two-fold: a typical workflow for
web servers (such as Apache, Nginx) \cite{apache-prefork-2.4, nginx-inside-performance-scale-2015} is to fork on receiving a request; the second is
that if a \ac{ta} could call fork() to remove the system call filter, our
solution would not be secure. Fork following is visualised in
Figure~\ref{fig:fork-follow-process-tree} with two process trees. Fork
following means each child process is protected by \af. Without it, children
are left unprotected.

\begin{figure}[ht]
    \centering
    \includegraphics[width=0.8 \linewidth]{./diagrams/fork-following.drawio.pdf} 
    \caption{A diagram showing two different process trees: one with fork
    following (left) and one without fork following (right).}
    \label{fig:fork-follow-process-tree}
\end{figure}


To do this, we can also check if the parent process's \ac{pid} is in the follow
map. If it is, we add the calling \ac{pid} to the follow map and continue with
the filtering. So, we only ignore the system call if \textit{neither the current nor
parent \ac{pid} exists in the follow map}.

Having decided to filter a system call, we need to identify where the system call was made in the process's
address space. We might use the value of the \ac{pc} when
the system call was made, but this turns out to be unhelpful.
Almost all system calls are made via \ac{libc} in a dynamically linked 
application, and so the \ac{pc} almost always points to \ac{libc} instead of the
library that actually made the system call.

\subsubsection{Finding the system call Site}\label{subsubsec:find_syscall}

Therefore, \af must find the \textit{first non-}\ac{libc} \textit{return pointer in the
user space stack} and treat this address as the true system call site. 
We trace through the sequence of function calls that led to the system
calls and find the first call that was not made by \ac{libc}.

Consider a ``Hello, World'' program in C (see Listing~\ref{lst:hello-world}). \texttt{printf()} is a 
function implemented in \ac{libc} which invokes the \texttt{write()} system call,
which calls the \texttt{syscall} x86
instruction\footnote{This is a simplified stack trace: in \texttt{musl} -- a
    ``simple'' \ac{libc} implementation --
\texttt{write()} calls a wrapper function \texttt{system call\_cp\_c()}. However,
the idea still holds.}. This means the user space stack may look like the stack
presented in Figure~\ref{fig:hello-world-stack}

\begin{figure}[h]
    \centering % Center the whole figure environment content

    % --- Subfigure for the Diagram (Left) ---
    \begin{subfigure}[b]{0.48\linewidth} % Adjust width as needed
        \centering % Center the image within the subfigure
        \includegraphics[width=\linewidth]{./diagrams/hello-world-stack.drawio.pdf}
        \caption{High-level \afg core filtering logic.} 
        \label{subfig:stack-diagram} % Sub-label for diagram
    \end{subfigure}
    \hfill % Add horizontal space between subfigures (pushes them apart)
    % --- Subfigure for the Code Listing (Right) ---
    \begin{subfigure}[b]{0.48\linewidth} % Adjust width as needed
        \centering % Center the code block if needed (often takes container width)
        % Use minted for the code listing
        \begin{minted}[fontsize=\small, frame=lines, framesep=2mm, breaklines]{c}
#include <stdio.h>

int main() {

    // printf @ 0x0000F722AE12
    printf("Hello, World!\n");    

    return 0;
}
        \end{minted}
        \caption{Hello World example in C.} % Sub-caption for code
        \label{subfig:hello-world-code} % Sub-label for code
    \end{subfigure}

    % --- Overall Figure Caption and Label ---
    \caption{Diagram illustrating an example call stack for a Hello, World
    program with corresponding code example.}% Overall caption
    \label{fig:hello-world-stack} % Overall label

\end{figure}

The addresses shown in the diagram are \textit{return pointers}: these are the
memory addresses of the instructions which added a new frame to the stack. In this
context, it is helpful to think of return pointers as the \textit{address of a
function}. Note also that the top few stack frames belong to \ac{libc}, with
the main binary a few frames deeper.

The idea is that here, we want to classify this system call as having been made from
the \texttt{main} binary, not from \ac{libc}. We can do this by
\textit{unwinding the stack} until we find a return pointer which is not within
\acg{libc} address space. This return pointer is then used as the system call
invocation site, as shown in Figure~\ref{fig:stack-unwinding}.

\begin{figure}[ht]
    \centering
    \includegraphics[width=0.95 \linewidth]{./diagrams/stack-unwinding.drawio.pdf} 
    \caption{A figure showing the stack unwinding process.}
    \label{fig:stack-unwinding}
\end{figure}

\subsubsection{Finding the Calling Shared Object File}\label{subsubsec:find_so}

Having found the memory address of the function which made the system call, we
can now figure out which shared library made the system call. This is a
question of checking the process's \ac{vma} to find which memory region the
function belongs to. 

Each shared library exists in the \ac{vma} as a \textit{file-backed memory region}.
In-kernel, the \ac{vma} is implemented as a \texttt{vm\_area\_struct} struct with a
(indirect) reference to the name of its backing file. Thus, finding which
shared library ``owns'' the system call site is a question of finding the
corresponding \texttt{vm\_area\_struct} struct and reading the filename.
Alternatively, this information can be read from user space via the
\texttt{/proc/PID/maps} pseudofile.

\begin{figure}[h]
    \centering
    \includegraphics[width=\linewidth]{./diagrams/core-filter-flowchart.drawio.pdf}
    \caption{A figure showing \afg core filtering logic at a high level.}
    \label{fig:core-filter-flowchart}
\end{figure}

\subsubsection{To Kill or not to Kill?}

Knowing which shared object file made the system call allows us to find the correct
system call whitelist and intervene if need be.

To do this, we keep a mapping from the shared library filename to a set of allowed
system calls. We use the previously calculated filename as a key to this map to
retrieve some (pre-configured) whitelists. If the system call number is in the
whitelist, no action is taken and the \ac{bpf} tracepoint returns with code 0.

If the system call is not on the whitelist, \af will intervene. The intervention
policy is user-configurable and has to be read from a config map. User space is
informed that a system call has tripped the filter as the offending \ac{pid} is
written to a ringbuffer. If the policy is set to kill the process (or all 
processes in the \ac{pid} filtering map), \af sends a kill signal to the 
offending \ac{pid} from \ac{bpf}. 

When user space reads \iac{pid} from the ringbuffer, it acts according to how its user-defined configuration. \af logs a warning in warn mode, but in ``kill all'' mode, sends a 
kill signal to each \ac{pid} in the filtering map. This has to be
done asynchronously (by user space) as \ac{bpf} only allows you to send signals
to the calling process. The pseudocode in
Listing~\ref{lst:syscall-filter} and diagram in 
Figure~\ref{fig:core-filter-flowchart} summarise the presented filtering logic.

This design is minimal, modular, and easy to reason about. We implement it with
with a single \ac{bpf} program running on the \texttt{raw\_tp/sys\_enter}
tracepoint with two core maps: one for storing the \acp{pid} of filtered apps
and the other for mapping shared library names to system call whitelists.

What we present here is the interesting part of the application
- the core filtering logic. The frontend is much larger (in terms of LoC) than
the \ac{bpf} program and also serves important functions in making the system
work.

\subsubsection{Configuring Whitelists and \ac{libc}}

The frontend is responsible for parsing the \af whitelists and loading them into
the whitelist map. We looked to build on existing standards for defining
system call whitelists but found none suitable for our use case. Seccomp filters,
for example, are defined as a \iac{bpf} program - this isn't user-friendly and
does not allow for mapping shared libraries to system call filters. Therefore, we defined our
own standard: a TOML file structured as in Listing~\ref{lst:toml-whitelist}.

The frontend is also responsible for supporting performance optimisations
discussed in Section \ref{subsubsec:impl-find-site-opt}. One of these optimisations is
to keep track of \acg{libc} address space in a map, as we assume \ac{libc} will not
change. The frontend is responsible for finding the filtered app's \ac{libc}
address range and storing it in a map before launch.

Ensuring that frontend functionality, such as parsing whitelists, is correct is
trivial with unit tests. Testing \ac{bpf}, however, is much more difficult.
Writing tests is technically possible but requires mocking a lot of key system
resources, crucially for us, the stack and the \ac{vma}. Having to mock core
functionality of a system often leads to brittle tests which are not effective -
therefore, we chose a different method of validation and evaluation.

\subsection{Validation and Evaluation}

We plan to validate correctness using the \acg{ltp} ``syscall'' test
suite to ensure \af does not disrupt normal system call behaviour, supplemented
by manual verification of simpler programs. The validation process is presented
in Section~\ref{subsec:validation}.

The evaluation focuses on quantifying the security benefit
(Section~\ref{subsec:security-eval}) by comparing privilege reduction
against seccomp using a defined metric, alongside measuring runtime
performance overhead (Section~\ref{subsec:perf-eval}) across a suite of
benchmarks relative to seccomp and no filter to understand costs and
limitations. This requires the development of supporting tools for analysis
and whitelist generation, discussed in 
Section~\ref{subsubsec:additional-tooling}.

\subsection{Additional Tooling}\label{subsubsec:additional-tooling}

We designed \texttt{afgen} to generate our whitelists. \texttt{afgen} works via dynamic analysis,
which is not a gold standard for whitelist generation \cite{OPTIMUS}. For our
use case, however, dynamic analysis is acceptable. Whitelist generation is orthogonal to our
work and is its own field of research. We have implemented a generator here
to aid development and evaluation, not to be used in production. More work is needed to create a
production-ready generator based on static analysis.

We implement the whitelist generator using a lot of the same functionalities of
\af. The key difference is that after identifying the system call site, we write a
system call number to the syscall site's map. When the program finishes executing
(or on Ctrl-C), the frontend reads the
contents of the \ac{bpf} whitelist map and marshals the results to a TOML file.
The pseudocode is provided in Listing~\ref{lst:generator-pseudocode} and shows
that the generation code heavily uses \afg functionality.

We also need a tool to launch an application with a seccomp filter enabled.
The tool takes as arguments an \af system call whitelist
and a path to an executable, parses the whitelist and generates a single
seccomp filter. This filter is made by taking the union of each shared 
library's allowed set of system calls in the \af whitelist. The filter can
then be applied to a benchmark to provide comparison data. This tool will be
referred to as \texttt{af-seccomp}.

The final additional tool we need is an evaluation tool, \texttt{syso}.
\texttt{syso} takes in a binary as input and compares the 
binary's privilege levels when \af and seccomp are applied. It also saves
some raw data to disc for further post-processing: a JSON mapping from
system call numbers to the number of times executed, broken down by library. An
example of this is available in Listing~\ref{lst:syso-data-dump}.

Having specified \af's core filtering behaviour and additional tooling
requirements, we move on to presenting \af's implementation.
