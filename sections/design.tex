\section{Design} \label{sec:design}

% The aim of design: write like the larousse. High level design for someone to
% take and improve upon. 
%
% Idea is to motivate that this is well designed and well thought through.
% Key design decisions:
        
% ------------------------
% Design introduction: what is the design philosophy? why?

\af is a complex systems level piece of software and therefore required a
careful design phase. In a systems security project, added code and complexity
represent more chances for vulnerabilities to appear. Therefore, it was
important to design \af to be \textbf{as minimal as possible} while fulfilling
the promise of the proposed fine-grained filtering.

To do this, we defined a \textbf{threat model} (\ref{subsec:threat-model}) and
let primary requirements (\ref{subsec:requirements}) follow. These primary
requirements are presented alongside \textit{corollary requirements}
(\ref{subsubsec:corollary-reqs}) and design assumptions
(\ref{subsec:assumptions}).

\afg architecture is visualised in Figure \ref{fig:arch-overview}. Each
section's function is explained in turn and key design decisions are justified.
The project's nature also requires some auxiliary tooling, which is discussed
after \afg design is presented.

\subsection{Threat Model}\label{subsec:threat-model}

% TODO: be more precise about <exploitation>
Our threat model lays out situations where \af is effective, and the 
pre-conditions that ensure applications being protected by \af are secure.

Firstly, we assume that 1) a \ac{ta} has gained \ac{rce} privileges on the host
machine \footnote{We refer to the machine running \af as the \textit{host
machine}}. That is, the \ac{ta} can execute any code they wish on the target
machine. This is one of the most serious breaches that can happen and often
lead to denial of service, data loss, or data theft.%TODO: cite RCE bad

The attacker aims to use their \ac{rce} to \textbf{escalate privilege}. This
means we assume that the \ac{ta} 2) doesn't have root privileges on the host,
but will act to get them.

% TODO: cite swc contains TA to compromised compt.
\af assumes that 3) some form of \textbf{software compartmentalisation} is in
place on the host machine. This compartmentalisation confines the \ac{ta} to the
compromised compartment. Importantly for \afss,~this will prevent the \ac{ta} 
from branching to an area of the address space that has a different allowed set
of system calls.

% Use/reference BPF threat model
% https://www.linuxfoundation.org/hubfs/eBPF/ControlPlane%20%E2%80%94%20eBPF%20Security%20Threat%20Model.pdf#page=11.14

\subsection{Requirements} \label{subsec:requirements}

Defining the high level requirements for \af is now possible, as we can reason
about exactly \textbf{what \af must protect against}.

\afg key feature is that it should detect where an application makes a 
\textbf{disallowed syscall} and intervene accordingly. 

This intervention should be \textbf{user configurable}: warning the user, killing the 
malicious process, be \textbf{user configurable}: warning the user, killing 
the malicious process, and killing all protected processes should all be
possible. This gives the user flexibility to choose how to trade-off
availability and data integrity/confidentiality on a per-application basis.

We should see a \textbf{significant reduction} in the set of syscalls a \ac{ta}
can access after comprimising an application, and we want to do this
\textbf{without detrimentally impacting performance}.

Preliminary results are promising on this front: nginx shows a 23.7\% in
privilege reduction when compared to seccomp, with an average slowdown of $x\%$ - 
This fulfils the requirement for an increase in security at reasonable runtime cost. 
A more detailed breakdown of results can be seen in the Evaluation (\ref{sec:evaluation})

To summarise 

\begin{enumerate}
    \item \af should \textbf{detect} map a syscall whitelist to a contiguous
        area of the \ac{vma}
    \item \af should detect \textbf{where} in the \ac{vma} a syscall was made
        from and \textbf{intervene} if the syscall isn't whitelisted
    \item \afg intervention policy should be \textbf{user configurable}: issuing
        a warning, killing the malicious process, and killing all processes
        being filtered should all be possible
    \item \af should provide a \texbf{significant level of privilege} reduction
        when compared to a regular seccomp filter at reasonable s
\end{enumerate}

\subsection{Corollary Requirements} \label{subsubsec:corollary-reqs}

\subsection{Assumptions}\label{subsec:assumptions}
\subsection{Architecture} \label{subsec:arch}

\begin{figure}[h]
    \centering
    \includegraphics[width=0.8 \linewidth]{./diagrams/TODO.pdf} 
    \caption{High level overview of \afg architecture}
    \label{fig:arch-overview}
\end{figure}

