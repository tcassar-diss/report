\section{Background}\label{sec:background}

\af is a project built on a complex stack of systems concepts and technologies
such as process stacks and address spaces, system calls, \ac{bpf}, and software
compartmentalisation. 

This chapter aims to give the reader the minimum background knowledge needed
to understand the motivation, design, and implementation of \af. We argue that
\af is novel and discuss how it  fits into the current research landscape in
Section~\ref{sec:related-works}.

\subsection{Linux Processes}

Processes are a key concept in both the Linux kernel and in system call
filtering. A process can be loosely thought of as an application - more 
concretely, it is an \textit{instance of an executing program} 
\cite{LINUX_PROGRAMMING_INTERFACE}. \af is applied to a process and requires 
access to the process's \ac{vma}, stack, and metadata (e.g. \ac{pid}) to 
function.

\subsubsection{Virtual Memory}

Each process accesses memory through the \textit{virtual memory} abstraction.
In the kernel, each processes has an \texttt{mm\_struct} which contains a
reference to several \texttt{vm\_area\_struct} structs. 
\texttt{vm\_area\_struct} structs contain information about a contiguous
range of memory addresses including permissions, relevant start and end
addresses, and, crucially for \af, the file that was used to populate these
memory addresses \cite{love2005linux}. These are called \textit{file backed
memory regions} and are visualised in Figure~\ref{fig:file-backed-vma}.

\begin{figure}[ht]
    \centering
    \includegraphics[width=0.8 \linewidth]{./diagrams/file-backed-vma.drawio.pdf} 
    \caption{A figure showing a file being loaded into a process's \ac{vma} to
    create a \textit{file backed memory region}. One vertical chunk represents
one address range represented by a \texttt{vm\_area\_struct} struct.}
    \label{fig:file-backed-vma}
\end{figure}

A common example of processes with file-backed regions in their address space
are dynamically linked executables. Dynamically linked executables contain
multiple file backed memory regions, with different regions being associated
with different shared libraries. Dynamically linked executables are mapped
into a processes address space by the linker at runtime, with the
\texttt{\_\_code\_\_} sections of each \ac{elf} binary are mapped contiguously 
in memory \cite{DYNAMIC_LINKERS_OS}. The \ac{vma} of a process can be
inspected from userspace either through the \texttt{pmap} command line tool, or
by reading the contents of the \texttt{/proc/pid/maps} file (where \texttt{pid}
is replaced by the relevant process's \ac{pid}.

This is how \af splits up applications for system call filtering: a system call
whitelist per file backed memory region.

\subsubsection{Stack Unwinding}

A section of each process's \ac{vma} is dedicated to the stack. The stack in
this context refers to an \acg{os} stack as opposed to the generic data
structure. Recall that a stack is used to manage function calls within a
process, with each function call pushing a stack frame and each return popping
said stack frame.
 
Debugging, observability, and security tools often inspect the process stack
while the program is running through a mechanism called \textit{stack
unwinding} \cite{gregg2014linux, kilroy2022linker}. This involves looking at 
each frame present in the userspace stack without popping any frames. We
introduce stack unwinding here as \af relies on stack unwinding for part of
its design: see Section~\ref{subsubsec:find_syscall}.

\subsection{Linux Security Mechanisms}

Processes, aside from being a convenient way for the \ac{os} to reason about
operations, also have security mechanisms inbuilt. Some of these such as
permission, \acp{uid}/\acp{gid}, and \ac{aslr} are relevant to \af.

\subsubsection{Process Permissions}

In Linux, each process is associated with \iac{uid} and one or more \acp{gid}.
These IDs determine which privileges different processes are allowed - these
privileges are enforced by the kernel. \af needs to be run with
\iac{uid}/\iac{gid} combination which gives the process the
\texttt{CAP\_SYS\_ADMIN} privilege in order to mount \ac{bpf} programs,
discussed further in Section~\ref{subsec:bgd-bpf}. However, \af needs to ensure
that the applications it launches with it's system call filter enabled does not
have access to this permission, as with it an attacker would be able to
circumvent the filter (as discussed in Section ~\ref{subsubsec:spawning-exec}).

Since \af needs to spawn processes, we need to take \ac{aslr} into
consideration. \ac{aslr} is a security feature of Linux that makes it harder for
attackers to exploit applications with \ac{rop} \cite{ASLR_GUARD}. When
enabled, the linker will change the addresses of the stack, heap, and file
backed address space regions in the new process's \ac{vma}. Crucially, \ac{aslr}
doesn't affect the address space of forks (as the linker is not involved in
a \texttt{fork()} system call). This allows us to run \af with full \ac{aslr}
enabled for our suite of benchmarks (see 
Section~\ref{subsec:benchmark-selection}), but is something to take into
consideration.  

The final built-in security feature that warrants mentioning here is the user
space/kernel space seperation. User space is designed to run untrusted 
applications. There is no direct access to hardware, I/O devices, or the 
network from userspace. Kernel space, on the other hand, is a privileged 
environment than userspace. The kernel has direct access to hardware, memory, and the
network. As such, one can do a lot of damage from kernel space including causing
the kernel to crash.

When an application needs to perform a privileged action, such as creating a
file, it must ask the kernel to do the action on the applications behalf. The
application does this by issuing a system call, and is visualised in
Figure~\ref{fig:syscall-flow}.

\begin{figure}[h]
\centering
\scalebox{0.75}{%
\includegraphics{./diagrams/syscall-flow.drawio.pdf}
}
\caption{A diagram showing a simple, generalised system call
flow}\label{fig:syscall-flow}
\end{figure}

Since a system call is a request for a privileged action, analysing system calls
is a well-established idea in systems security. Host-based intrusion detection
systems such as \cite{10.1145/3214304} use system call invocation patterns to look for
signs of compromise.

\subsection{System Calls in Existing Security Mechanisms}

Seccomp is a system call filter that has been part of the Linux kernel since
2005 \cite{arcangeli_seccomp_2005}. Seccomp's goal is to restrict the set of
system calls that a process can make, thereby reducing it's privilege and making
any attackers who may have compromised the process less able to cause damage.

Despite initial skepticism and slow adoption
\cite{TORVALDS_ANYONE_USES_SECCOMP}, seccomp is widespread today. Android has
used seccomp since 8.0 Oreo, Docker containers are
created with a seccomp filter by default, and FireFox uses it as a means of
sandboxing content processes \cite{android_seccomp_oreo, docker_seccomp,
firefox_seccomp}.

Seccomp is invoked with the \texttt{prctl} system call in one of two modes: strict
or filter. In strict mode, only three system calls are allowed: \texttt{read},
\texttt{write}, and \texttt{exit} \cite{MAN_PAGES_SECCOMP}. This is typically
too strict to be helpful and so the filter mode also exists.

In filter mode, a \ac{bpf} program (plus some metadata) is supplied in the arguments to
\texttt{prctl}. When a system call occurs, this program is run and its logic
determines how to handle the system call. 

Seccomp's man pages entry suggest that it may be able to provide fine-grained
system call filtering based on the process's address space, which is what we
are proposing for this project. It's entry states that seccompBPF programs have
access to the ``\texttt{instruction\_pointer} \textit{field}'', which
``\textit{might be useful in conjunction with the use of /proc/pid/maps to
perform checks on which region\dots made the system call}''. As noted in
\textcite{yang2024makingsyscallprivilegeright}, seccompBPF does not provide
access to anything other than ptrace registers, which makes it too restrictive
for \af.

As we argue in Section~\ref{subsubsec:find_syscall}, the instruction pointer
address alone is insufficient for the fine-grained filtering mechanism we
propose here. Thus, a novel approach built on different technology is needed
to implement \af. 

\subsection{The (Extended) Berkley Packet Filter}\label{subsec:bgd-bpf}

We decided to use \ac{bpf} in our implementation of \af. \ac{bpf} is a technology
that allows developers to write custom programs which can be \textit{dynamically
loaded} into the kernel to alter the kernel's behaviour \cite{LEARNING_EBPF}.

Writing kernel code without \ac{bpf} is error-prone and typically
requires linking to the kernel \cite{UNDERSTANDING_LINUX_KERNEL}. In contrast,
\ac{bpf} programs can be configured to run on many instrumentation points, such
as a system call being triggered. \ac{bpf} also provides a \textit{verifier} which
ensures the  loaded program is safe \cite{LEARNING_EBPF}. This makes it
impossible to cause a kernel panic (or any similarly catastrophic event) when
writing \ac{bpf} although real-world exploits targeting the verifier have
been seen \cite{BPF_VERIFIER_EXPLOIT}.  

\subsubsection{Context and Use Cases}

As such, \ac{bpf} is used a lot in industry for observability tooling: being able to
dynamically load \ac{bpf} programs allows developers to instrument running processes
without changing any configuration files or using sidecar containers \cite{SIDECAR}.
\ac{bpf} is also often used in networking and firewalls 
\cite{LINUX_NETWORKING_OBSERVABILITY} - a \ac{bpf} program can be loaded 
onto the \ac{xdp} to inspect packets at the 
``\textit{earliest possible point in the network driver}'' 
\cite{CILIUM_BPF_XDP_INTRO}. 

The trade-off here is that \ac{bpf} is extremely limited in expressivity:
something as mundane as reading every element of an array can be impossible due
to the verifier needing to ensure that every memory access will not result in 
a null dereference (even though a loop helper now exists 
\cite{BPF_LOOP_COMMIT}). Lots of accesses to structs must be done through 
\textit{helper functions} - convenient when they exist, but sometimes mean a trivial
task is impossible (without writing custom kernel code) \cite{bpf_kfuncs_docs}. 

Despite being originally designed as observability tooling, \ac{bpf} is
increasingly being used for security applications such as mitigating SPECTRE
\cite{SPECTRE_BPF_MITIGATION} or dropping carefully-crafted packets that can
lead to an exploit in kernel \cite{BPF_PACKET_MITIGATION}. Integration with \ac
{lsm} has been added as part of a suite of security tools for \ac{bpf} 
\cite{StarovoitovBPFSecurity}, which shows there is precedent for using \ac{bpf}
as security tooling. Furthermore, seccompBPF is an extension of the existing
seccomp which can define a custom  system call filter in terms of a \ac{bpf} program
\cite{seccompBPF}.  Knowing this information justifies why developing 
\af in \ac{bpf} was a sensible idea.

\subsubsection{Structure of \iac{bpf} Program}

\Iac{bpf} program operates by attaching to specific kernel instrumentation
points (like tracepoints or function entries); it runs automatically
whenever kernel execution hits that point. Persistent state or configuration
is stored in \ac{bpf} maps, versatile key-value stores often accessible from both
the kernel program and user space applications. For efficiently sending
event data from the kernel to user space, mechanisms like ringbuffers or
perf buffers are used, allowing near real-time communication.

\af makes use of all of these \ac{bpf} concepts except perf buffers.
Section~\ref{sec:implementation} goes into further detail about specifics,
discussing which instrumentation point we attached our program to and why. We
also discuss which maps are used and where/why we use ringbuffers.

\subsection{Software Compartmentalisation}

The core idea of \af{}- decomposing the process address space to apply 
fine-grained system call restrictions - is an application of \textit{software
compartmentalisation}.

Software compartmentalisation is a form of defensive programming where an application
is broken down into multiple isolated compartments \cite{SOK}. Compartments
communicate only over secure interfaces and do not trust each other. An
attacker who manages to compromise one compartment will not be able to gain
access to data or resources assigned to another compartment.  

Implementing this form of compartmentalisation requires deep kernel
integration, making \ac{bpf}, with the capabilities discussed previously, a 
suitable tool for the task.  

Compartmentalisation has proven useful in containing memory safety issues
\cite{CONFFUZZ}, sandboxing untrusted third parties \cite{ANDROID_SOK} or
unsafe parts of languages \cite{MPK} (e.g. \texttt{unsafe} in Rust 
\cite{rustbook_unsafe}). Compartmentalisation is also an application of the
principle of least privilege to software \cite{PRIVMAN}. As discussed in 
Section~\ref{subsec:assumptions}, we also assume a compartmentalisation
mechanism is in place on our host system.
