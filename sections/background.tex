\section{Background} \label{sec:background}

%%% WHAT DOES THE BACKGROUND SECTION ACCOMPLISH?
% - Here is what you need to know
% - This is why it's relevant

% Aim for 2,000 words

\texttt{addrfilter} is built on a complex stack of systems technologies,
including Linux syscall mechanisms, compartmentalisation, and \ac{bpf}. This
chapter covers the necessary background.

\subsection{Linux Primitives}

There are two core ideas in Linux which underpin not only the design of
\texttt{addrfilter}, but also the ideas presented in this chapter. A refresher
of the \textbf{process} and \textbf{kernel} is provided here.

\subsubsection{Processes}

At the core of system call filtering is the concept of a \textbf{process}. A
process can be loosely thought of as an application - more concretely, it is an
\textit{instance of an executing program} \cite{LINUX_PROGRAMMING_INTERFACE}.

% Mini paragraph on the stack
    % Process has its own stack; used to manage functions and control flow
    % Stack is composed of _stack frames_; include **return pointers**
    % To analyse the code path which resulted in a system call, we can 
    % **unwind the stack**

% The return pointers of the stack are virtual memory addresses, pointing to
% some space in the process's vma (.text section is loaded at runtime by the
% loader). 

% In a dynamically linked executable, shared libraries will be mapped
% into different regions of memory. They are file backed, and their originating
% file is stored in the `vma_struct`. By inspecting the process's virtual
% memory, it is possible to identify the shared library file which triggered a
% system call.

%TODO:
% Diagram visualising stack and undwinding with return pointers
% Diagram visualising process's VMA

\subsubsection{The Linux Kernel}
Operating systems have two modes, or \textit{worlds}, associated with them: user
mode and kernel mode. User mode, or userspace, is designed to run untrusted 
applications. There is no direct access to hardware, I/O devices, or the 
network from userspace.

Kernel mode, or kernelspace, on the other hand, is a more privileged environment
than userspace. The kernel has direct access to hardware, memory, and the
network. As such, one can do a lot of damage from kernel space including causing
a \textbf{kernel panic} \cite{panic9}.
% TODO: Explain a kernel panic or reword to a crash owtte

Therefore, it is vital that only trusted code is run in kernel mode, and that
non-trusted code is only ever run in user mode. This begs the question: if
applications aren't allowed direct access to the hardware, then how do user
applications, such as web browsers, interact with hardware resources? The answer
is via \textbf{system calls}.

\subsection{System Calls}

When an application running in userspace wants to access hardware - for example,
to request a page in memory - it issues a \textbf{system call} (or syscall). System calls
form an \ac{API} which allows userspace programs to ask the kernel to complete some
task, and return the result back to userspace.

In the case of requesting a memory page, the userspace application will make a
call to \texttt{mmap()} from the \ac{OS}'s C standard library (e.g. \texttt{libc}).
This triggers a \textit{world switch}: the process of changing the hardware
to/from kernel mode.

A world switch is a costly operation, involving flushing any virtually mapped
caches and copying all the contents of all \ac{CPU} registers to the user stack. This
is done to ensure a strict separation between kernelspace and userspace, and to
make sure that the userspace process can continue running as usual after the
syscall.

Syscall invocation patterns can serve as indicators of compromise
and are a common feature of host based intrusion detection systems \cite{10.1145/3214304}. 
Applications only need to make a subset of syscalls
- if an app makes a syscall that is not in its source code, that is a clear
indication of compromise. Existing software such as \texttt{seccomp} uses this
to create a \textbf{syscall filter}, killing any processes which make
blacklisted syscalls. 

\texttt{addrfilter} builds on what \texttt{seccomp} does by breaking down the
process's address space into smaller subsections, and assigning a bespoke filter
to each subsection. This design is a form of \textbf{software
compartmentalisation}. 

\subsection{Software Compartmentalisation}

Software compartmentalisation is a form of defensive programming where an
application is broken down into multiple isolated compartments \cite{SOK}. Compartments
communicate only over secure interfaces and do not trust each other. An attacker
who manages to compromise one compartment will not be able to gain access to
data or resources assigned to another compartment. 

Compartmentalisation has proven useful in containing memory safety issues
\cite{CONFFUZZ},
sandboxing untrusted third parties \cite{ANDROID_SOK} or unsafe parts of
languages \cite{MPK} (e.g.
\texttt{unsafe} in Rust \cite{rustbook_unsafe}). Compartmentalisation is also an
application of the principle of least privilege to software \cite{PRIVMAN}.

\texttt{addrfilter} is a work under the umbrella of software
compartmentalisation for this reason: it decomposes the process address space
and restricts which separated units are allowed to make specific syscalls.
This functionality is not trivially supported by common tools used to work with
syscalls such as \texttt{strace}. As such, implementing \texttt{addrfilter}
required writing extensive amounts of kernel code; \ac{bpf} was chosen for this
task.

\subsection{Introducing eBPF}

\ac{bpf} is a technology that allows developers to write custom programs which
can be \textit{dynamically loaded} into the kernel to alter the kernel's
behaviour \cite{LEARNING_EBPF}. 

Writing kernel code without \ac{bpf} is error-prone and typically
requires linking to the kernel \cite{UNDERSTANDING_LINUX_KERNEL}. In contrast,
\ac{bpf} programs can be configured to run on many instrumentation points, such
as a syscall being triggered. \ac{bpf} also provides a \textbf{verifier} which ensures the  loaded program is safe \cite{LEARNING_EBPF}.
This makes it impossible to cause a kernel panic (or any similarly
catastrophic event) when writing \ac{bpf} although real-world exploits targeting the
verifier have been seen \cite{BPF_VERIFIER_EXPLOIT}. 

As such, \ac{bpf} is used a lot in industry for observability tooling: being able to
dynamically load \ac{bpf} programs allows developers to instrument running processes
without changing any configuration files or using sidecar containers \cite{SIDECAR}.
\ac{bpf} is also often used in networking and firewalls \cite{LINUX_NETWORKING_OBSERVABILITY} - a \ac{bpf} program can
be loaded onto the \ac{xdp} to inspect packets at the ``\textit{earliest possible
point in the network driver}'' \cite{CILIUM_BPF_XDP_INTRO}. 

The trade-off here is that \ac{bpf} is extremely limited in expressivity:
something as mundane as reading every element of an array can be impossible due
to the verifier needing to ensure that every memory access will not result in 
a null dereference (even though a loop helper now exists \cite{BPF_LOOP_COMMIT}). Lots of accesses to structs must be done through
\textbf{helper functions} - convenient when they exist, but sometimes mean a
trivial task is impossible (without writing custom kernel code) \cite{bpf_kfuncs_docs}.

Despite being designed as observability tooling, \ac{bpf} is increasingly being
used for security applications such as mitigating SPECTRE
\cite{SPECTRE_BPF_MITIGATION} or dropping carefully-crafted packets that can
lead to an exploit in kernel \cite{BPF_PACKET_MITIGATION}. Integration with
\ac{lsm} has been added as part of a suite of security tools for \ac{bpf}
\cite{StarovoitovBPFSecurity}, which
shows there is precedent for using \ac{bpf} as security tooling. Furthermore,
seccompBPF is an extension of the existing seccomp which can define a custom
syscall filter in terms of a \ac{bpf} program \cite{seccompBPF}. Knowing this information
justifies why developing \texttt{addrfilter} in \ac{bpf} was a sensible idea.

The next section explores more about seccomp and traditional syscall
filtering mechanisms, and discusses the gaps in the security guarantees they
provide.

\subsection{Traditional Filtering Mechanisms}

Syscall filtering was added to the Linux Kernel in 2005 for use in grid
computing - a paradigm involving running untrusted, compute bound programs on a
personal computer \cite{GRID_COMPUTING_INTRO}. The
filtering mechanism was called \texttt{seccomp}, and its purpose was to
restrict the set of syscalls that a process could make \cite{arcangeli_seccomp_2005}.

Despite slow initial adoption, with Linus Torvalds famously questioning whether
anyone actually uses seccomp in 2009 \cite{TORVALDS_ANYONE_USES_SECCOMP}, it is
widespread today. Android has used it since 8.0 Oreo, Docker containers are
created with a seccomp filter by default, and FireFox uses it as a means of
sandboxing content processes \cite{android_seccomp_oreo, docker_seccomp,
firefox_seccomp}.

Seccomp is invoked with the \texttt{prctl} syscall in one of two modes: strict
or filter. In strict mode, only three system calls are allowed: \texttt{read},
\texttt{write}, and \texttt{exit} \cite{MAN_PAGES_SECCOMP}. This is typically
too strict to be helpful and so the filter mode also exists.

In filter mode, a \ac{bpf} program (plus some metadata) is supplied in the arguments to
\texttt{prctl}. When a system call occurs, this program is run and its logic
determines how to handle the system call. 

Seccomp also provides functionality which looks like it might be useful for use in
\texttt{addrfilter}. From the man pages:

\begin{quote}
    The instruction\_pointer field provides the address of the machine-
    language instruction that performed the system call.  This might
    be useful in conjunction with the use of /proc/pid/maps to perform
    checks based on which region (mapping) of the program made the
    system call.  (Probably, it is wise to lock down the mmap(2) and
    mprotect(2) system calls to prevent the program from subverting
    such checks.)
\end{quote}

However, the instruction pointer alone is insufficient for \texttt{addrfilter}'s
requirements. 
% TODO: Cite initial eval run results? Include plot here?
Syscalls are almost always made from \texttt{libc}, which means the instruction
pointer on syscall entry almost always originates in \texttt{libc}. This doesn't
provide any useful information on where the system call was made. 

Furthermore, relying on \texttt{/proc/pid/maps} requires added engineering
effort to avoid \ac{toctou} complications -- clearly, a new solution is needed
to be able to filter system calls by address space. 

Since seccompBPF doesn't provide all the functionality needed for 
\texttt{addrfilter}, and is more restrictive than (the already restrictive) 
\ac{bpf}, it isn't the right tool for implementing the fine-grained system
call filtering proposed.

To do this, a novel approach is needed. Kernel code will require access to the full
\ac{bpf} featureset involving interaction with various process-specific data
structures (e.g. the process's stack, virtual address space, and filesystem).

\subsection{Conclusion}

This chapter has covered background essential to understanding the need for, and
the design choices behind, \texttt{addrfilter}. Additional information on the
virtual address space, stack, and filesystem of a process was given. System
calls were introduced as a means of communication between kernel- and userspace.
Software compartmentalisation and its relevance to \texttt{addrfilter} was
discussed, and \ac{bpf} was motivated as the method of choice for writing kernel
code. 

The next section will combine all of this relevant prerequisite knowledge into a
high level design for \texttt{addrfilter}.
