\section{Related Works}\label{sec:related-works}

Having detailed how \af works and quantified its security and performance
impacts on a diverse set of benchmarks, we turn our attention to discussing how
\af fits into the wider system call literature.

\subsection{History of System Call Filtering}

System call filtering first appeared in literature in the 1990s with seminal
works like \textcite{wagner1999janus}, and predominantly implemented filtering
logic in user space \cite{somayaji2000automated, garfinkel2004ostia}. 
The world-switching associated with these user space solutions made
system call filtering too slow for production use.

In 2005, Andrea Arcangeli created seccomp -- an in-kernel system call filtering
solution. Seccomp was added to the Linux that year, but
saw slow adoption with Linus Torvalds asking if anyone used seccomp in
2009 \cite{arcangeli_seccomp_2005, TORVALDS_ANYONE_USES_SECCOMP}. 

In the past 16 years, system call filtering has become increasingly common.
Seccomp is the industry standard and is now widely supported. Chromium
\cite{chromium_sandboxing}, OpenSSH \cite{openssh60}, Docker \cite{docker_seccomp}, Android \cite{android_seccomp_oreo}, Package
Managers \cite{flatpak_seccomp_profile}, and emulation tooling all run with
seccomp filters by default.

\subsection{Fine-Grained Filtering Research}

This widespread adoption means that system call filtering is still a widely
researched topic in modern systems security literature. 

A trend is emerging in system call filtering research towards 
\textit{finer-grained} system call filters, in which multiple smaller, more
precise filters are applied to an application 
\cite{SYSPART, SYSXCHG, ahad2023freepart}. The vast majority of this work is
\textit{temporal}: different filters are applied to the same application at
different points in time \cite{SYSPART, TEMPORAL_SPEC}. Temporal syscall
filters rely on accurate phase detection, and any misidentification or delay
switching filters can create brief but exploitable windows where security
policies are not fully enforced \cite{TEMPORAL_SPEC}. 

SysPart \cite{SYSPART} heavily inspired \af as it also
looks at fine-grained system call filtering. SysPart focuses on \textit{temporal
separation} - an application is split into two phases, setup and steady-state.
\af, in contrast, defines \textit {spatial separation}, applying different
filters to different parts of the address space. Temporal separation is an
idea which predates SysPart, proposed by \textcite{TEMPORAL_SPEC}, who note that
the temporal filtering scope is primarily limited to servers as server
applications often exhibit different startup and steady-state behaviour. 

FreePart \cite{ahad2023freepart} is another isolation tool that applies
system call filters based on temporal specialisation. It observes that privileges, including system calls, are designed for data processing applications
and often align with the specific \acp{api} (e.g. TensorFlow
\cite{tensorflow2015}, OpenCV \cite{opencv_library}) used during execution. While
conceptually similar to \af, FreePart differs in two ways: its scope is broader
than just system call filtering, and it dynamically changes filters based on
the program's execution phase. \af, however, is a spatially scoped filtering
model, applying distinct system call whitelists to file-backed memory regions.
We enable per-library privilege enforcement rather than relying on execution 
phase.

SysXCHG \cite{SYSXCHG} also implements fine-grained filtering by changing
filters at runtime. SysXCHG does not use a temporal separation model and instead
focuses on subprocesses. Seccomp (and \af) \textit{pass down} filters from
parent to child processes, meaning whitelists must be permissive enough
to cover both the parent and child processes. SysXCHG replaces the child
process's filter, allowing it to be \textit{more precise} than a filter which
must cover both processes. SysXCHG solves a different problem to \af -- one of
privilege inheritance across \texttt{execve()} calls.

\subsection{Whitelist Generation}

While orthogonal to our work, whitelist generation remains a key part of
effective system call filtering solutions. Recall that we used \texttt{afgen} --
a dynamic analysis-based tool -- to generate \af whitelists, as no current
tooling can generate per-shared library whitelists. This is a limitation
of our current approach, so we investigate some work in static analysis
whitelist generation.

Dynamic analysis whitelist generation is a limitation because dynamic analysis
cannot find all possible system calls that an application may
make. This may lead to a false positive at runtime, and a production application
may be killed illegitimately \cite{XING2022105}. 

BSide \cite{BSIDE} uses static analysis to generate a precise
superset of the system calls an application may need to make. It is built on 
the Angr \cite{angr2025} binary analysis framework and uses symbolic execution
to determine possible register values when a system call is issued. In future,
building on BSide to create a more robust \af-specific system call whitelist
generator will make \af more production-ready.

\subsection{Software Compartmentalisation}

Software compartmentalisation is a form of defensive programming where an
application is decomposed into different \textit{compartments}, each isolated
from one another \cite{SOK}. Furthermore, compartments communicate over
well-defined, secure interfaces \cite{CONFFUZZ}. Therefore, attackers who
compromise a compartment are confined to it, limiting the damage they can
cause to the overall system.

As referenced in Section~\ref{subsec:objs-assumpions-tm}, \af presupposes that some
form of compartmentalisation, which will prevent an attacker from branching to
another place in the process's \ac{vma} as is in place. The CHERI capability
model \cite{CHERI} is a \textit{capability} model which extends the 64-bit MIPS
\ac{isa}\footnote{This means that it is incompatible with \af as \af
currently only supports the x86 \ac{isa}.}, and can stop an attacker from moving through the
\ac{vma} at will. 

System call filtering has been applied to \ac{pku} based applications as with
Jenny \cite{JENNY}. Jenny is a system call filtering tool that addresses
the security concerns with system calls issued from
\ac{pku}-protected processes. \acp{pku} are a hardware isolation primitive that
effectively provide \textit{data sandboxing} (but do not prevent attackers
from executing code from other parts of the address space, making them unsuitable for
\af).

Finally, it is worth noting that \af itself falls under the umbrella of software
compartmentalisation research, as \af restrits the set of system calls a process
can make as per the \ac{polp} \cite{SALTZER_SCHROEDER}.

\subsection{\ac{bpf} in Security}\label{subsec:bpf-in-security}

Despite being originally designed as observability tooling, \ac{bpf} is
increasingly being used for security applications such as mitigating SPECTRE
\cite{SPECTRE_BPF_MITIGATION} or dropping carefully-crafted packets that can
lead to an exploit in kernel \cite{BPF_PACKET_MITIGATION}. Integration with \ac
{lsm} has been added as part of a suite of security tools for \ac{bpf} 
\cite{StarovoitovBPFSecurity}, which shows there is precedent for using \ac{bpf}
as security tooling. 

seccompBPF  is an extension of the existing seccomp, which can define a custom
system call filter in terms of a \ac {bpf} program \cite{seccompBPF,
jia2023programmablesecurityebpf}. Modern \ac{bpf} specific security frameworks 
such as Cilium \cite{cilium2025}, Falco\cite{falco2025}, and Tetragon 
\cite{tetragon2025} are production examples of security based \ac{bpf} tooling.

Thus, we show that \af fits into a niche gap in existing system call literature
and is genuinely a novel solution. \af relates to research in software
compartmentalisation as well as system call filtering, and its technology stack
is well precedented in systems security.
